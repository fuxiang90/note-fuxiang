\documentclass[9pt]{article}
\usepackage{fullpage}
\usepackage{amsmath}
\usepackage{amssymb}
\usepackage[usenames]{color}
\usepackage{CJKutf8} %处理中文
\usepackage[colorlinks,
            linkcolor=red,
            anchorcolor=blue,
            citecolor=green
            ]{hyperref}


\leftmargin=0.25in
\oddsidemargin=0.25in
\textwidth=6.0in
\topmargin=-0.75in
\textheight=9.25in

\raggedright

\pagenumbering{arabic}

\def\bull{\vrule height 0.8ex width .7ex depth -.1ex }
% DEFINITIONS FOR RESUME

\newenvironment{changemargin}[2]{%
  \begin{list}{}{%
    \setlength{\topsep}{0pt}%
    \setlength{\leftmargin}{#1}%
    \setlength{\rightmargin}{#2}%
    \setlength{\listparindent}{\parindent}%
    \setlength{\itemindent}{\parindent}%
    \setlength{\parsep}{\parskip}%
  }%
  \item[]}{\end{list}
}

\newcommand{\lineover}{
	\begin{changemargin}{-0.05in}{-0.05in}
		\vspace*{-8pt}
		\hrulefill \\
		\vspace*{-2pt}
	\end{changemargin}
}

\newcommand{\header}[1]{
	\begin{changemargin}{-0.5in}{-0.5in}
		\scshape{#1}\\
  	\lineover
	\end{changemargin}
}

\newcommand{\contact}[4]{
	\begin{changemargin}{-0.5in}{-0.5in}
		\begin{center}
			{\Large \scshape {#1}}\\ \smallskip
			{#2}\\ \smallskip 
			{#3}\\ \smallskip
			{#4}\smallskip
		\end{center}
	\end{changemargin}
}

\newenvironment{body} {
	\vspace*{-16pt}
	\begin{changemargin}{-0.25in}{-0.5in}
  }	
	{\end{changemargin}
}	

\newcommand{\school}[4]{
	\textbf{#1} \hfill \emph{#2\\}
	#3\\ 
	#4\\
}

% END RESUME DEFINITIONS

\begin{document}

\begin{CJK}{UTF8}{gkai}
%%%%%%%%%%%%%%%%%%%%%%%%%%%%%%%%%%%%%%%%%%%%%%%%%%%%%%%%%%%%%%%%%%%%%%%%%%%%%%%%
% Name
\contact{ 付 翔 }{邮箱: fuxiang90@gmail.com   电话: 18810542862 }{技术博客: \url{www.fuxiang90.com}  开源: \url{github.com/fuxiang90 } }


%%%%%%%%%%%%%%%%%%%%%%%%%%%%%%%%%%%%%%%%%%%%%%%%%%%%%%%%%%%%%%%%%%%%%%%%%%%%%%%%
% Objective
\header{自我介绍}

\begin{body}
	\vspace{14pt}
	男,热爱编程,对机器学习,搜索引擎和算法有浓厚的兴趣;有很强的自学能力,喜欢探索新事物.开源精神的支持和实践者.\\
        %热爱并实践开源精神,github地址 github.com/fuxiang90.
\end{body}

\smallskip


%%%%%%%%%%%%%%%%%%%%%%%%%%%%%%%%%%%%%%%%%%%%%%%%%%%%%%%%%%%%%%%%%%%%%%%%%%%%%%%%
% Education
\header{教育背景}

\begin{body}
	\vspace{14pt}
	\emph{北京邮电大学计算机院 计算机专业 } \hfill 2011/09-2014/03{} \\
% 之后再增加本科学历
  \medskip
	\emph{湖南农业大学经济学院 农业经济管理专业 } \hfill 2007/09-2011/6{} \\
\end{body}

\smallskip

%%%%%%%%%%%%%%%%%%%%%%%%%%%%%%%%%%%%%%%%%%%%%%%%%%%%%%%%%%%%%%%%%%%%%%%%%%%%%%%%
% Experience
\header{项目经验}

\begin{body}
	\vspace{14pt}
	
%%%%%%%%	
	
	\emph{阿里巴巴一淘搜索技术部 搜索研发实习生 }\hfill  {2013/7/15-至今}\\	
	\vspace*{-4pt}
	\begin{itemize} \itemsep -0pt  % reduce space between items
		\item  产品库聚类,去重
		%\item 2013年3月至四月,用c重写道路匹配的代码,内存减少500M ,速度提升50\%

	\end{itemize}



	%\emph{用c重写代码,内存由八百M减至三百M,速度提升50% }\hfill  {2013/3-2013/4}\\

%%%%%%%%%%%%%%

	\emph{广州及周边六省高快速路路况分析.}\hfill  {2012/6-2012/12}\\
	\vspace*{-4pt}
	\begin{itemize} \itemsep -0pt  % reduce space between items
		\item{\textbf{项目描述:}}{}  根据车辆的 GPS 信息,得到广州和周边六省的高快速路路况,为政府监管提供数据支持
		\item {\textbf{职责:}}{}  设计地图删减算法;实现 Astar 最短路径算法,替换原有的寻路算法;独立设计并实现基于权重和路网拓扑的地图匹配算法.
		%\item 2013年3月至四月,用c重写道路匹配的代码,内存减少500M ,速度提升50\%

	\end{itemize}

	%\emph{用c重写代码,内存由八百M减至三百M,速度提升50% }\hfill  {2013/3-2013/4}\\

%%%%%%%%%%%%%%

	
	\emph{基于历史数据的道路速度预测 .}\hfill  2013/3-2013/6\\
	\vspace*{-4pt}
	\begin{itemize} \itemsep -0pt  % reduce space between items
		\item{\textbf{项目描述:}}{}  广州市发布道路两个月历史速度信息(约2亿条),根据当前道路速度预测之后一段时间内的速度
		\item 设计基于有向图的调度算法,使在内存中道路信息最少,减少内存使用
		\item 用梯度下降算法处理道路的历史信息,生成预测模型
	\end{itemize}

%%%%%%%%%%%%%%%

 	\emph{参加雅虎北京黑客马拉松决赛,编程24小时 完成一个基于flickr的图片推荐的应用.} \hfill {2013/5/18 –2013/5/19}\\
	\vspace*{4pt}
%%%%%%%%%%%%%

%	 \emph{基于北邮 bbs 的检索系统的开发,个人开源项目,python开发.} \hfill {2012/8 –2012/12}\\
%	\vspace*{-4pt}
%	\begin{itemize} \itemsep -0pt
%		\item  单线程爬虫,定时爬取论坛信息
%		\item  单次遍历文档建立索引 ,平均每 s 处理 166 个文档,支持简单的查询,并可定制邮件推送服务
%		\item Redis 作为缓存服务,zeromq 处理 web 请求,Json 作为数据交互
%	\end{itemize}
% 	

% 	\emph{基于 sift 和 bag of word 图像分类.} \hfill {2013/1-2013/1}\\
% 	\vspace*{4pt}
%	\begin{itemize} \itemsep -0pt
%		\item  Sift 提取特征,然后聚类,训练数据向聚类中心映射,得到每一个训练数据在该聚类中心空间的一个低维表示
%		\item 开源软件 dlib 对训练集进行训练,利用svm 进行分类 
%	\end{itemize}
	
	
%	\emph{文本分类.} \hfill {2012/8 - 2012/12}\\
%	\vspace*{-4pt}
%	\begin{itemize} \itemsep -0pt
%		\item  {\textbf{项目描述:}}{}  数据挖掘课程期末大作业,贝叶斯实现分类算法,对文本进行分类,每个分类 1000 文本作为训练集
%		\item {\textbf{环境: }}{} linux ,c++ ,分词使用开源分词组件 ,取得了 84\%的正确率
%		%\item 程序得分最高,速度最快 ,取得了 84\%的正确率
%	\end{itemize}
\end{body}

\smallskip

%%%%%%%%%%%%%%%%%%%%%%%%%%%%%%%%%%%%%%%%%%%%%%%%%%%%%%%%%%%%%%%%%%%%%%%%%%%%%%%%
% search
\header{搜索实践}

\begin{body}
	\vspace{14pt}
	
 	\emph{基于北邮 bbs 的检索系统的开发,个人开源项目,python开发.} \hfill {2012/8 –2012/12}\\
	\vspace*{-4pt}
	\begin{itemize} \itemsep -0pt
%		\item  单线程爬虫,定时爬取论坛信息
		\item  单次遍历文档建立索引 ,平均每 s 处理 166 个文档,支持简单的查询,并可定制邮件推送服务
		\item 单线程爬虫, Redis 作为缓存服务,zeromq 处理 web 请求,Json 作为数据交互
	\end{itemize}
	
	\emph{小型网站关键词推荐 ,python,c++开发.} \\
	\vspace*{-4pt}
	\begin{itemize} \itemsep -0pt

		\item {\textbf{职责:}}{}   中期代码重构,串通整个流程,根据指定的关键词,给出和网站最相关的关键词
	\end{itemize}
	
 	\emph{基于 sift 和 bag of word 图像分类.} \hfill {2013/1-2013/1}\\
 	\vspace*{-4pt}
	\begin{itemize} \itemsep -0pt
		\item  Sift 提取特征,然后聚类,训练数据向聚类中心映射,得到每一个训练数据在该聚类中心空间的一个低维表示
		\item 开源软件 dlib 对训练集进行训练,利用svm 进行分类 
	\end{itemize}
	\emph{贝叶斯文本分类. linux ,c++ 实现} \hfill {2012/8 - 2012/12}\\
	\vspace*{-4pt}
	\begin{itemize} \itemsep -0pt
%		\item  {\textbf{项目描述:}}{}  数据挖掘课程期末大作业,贝叶斯实现分类算法,对文本进行分类,每个分类 1000 文本作为训练集
		\item {} 五个分类,每个分类 1000 文本作为训练集,分词使用开源分词组件 ,取得了 84\%的正确率
		%\item 程序得分最高,速度最快 ,取得了 84\%的正确率
	\end{itemize}


\end{body}

\smallskip

%%%%%%%%%%%%%%%%%%%%%%%%%%%%%%%%%%%%%%%%%%%%%%%%%%%%%%%%%%%%%%%%%%%%%%%%%%%%%%%%
% Skills
\header{Skills}

\begin{body}
	\vspace{14pt}
	\emph{\textbf{编程语言:}}{} 六万行C代码,熟悉C++和 python \\
%	\medskip
%	\emph{\textbf{个人开源代码:}}{} CSimpleLib ,fSearch-mini \\
	\medskip
	\emph{\textbf{操作系统:}}{} 3年Linux使用经验,Windows \\
	\medskip
	\emph{\textbf{编程工具:}}{} Vim,Codeblocks,Eclipse,Visual Studio,Git \\
\end{body}

\smallskip



%%%%%%%%%%%%%%%%%%%%%%%%%%%%%%%%%%%%%%%%%%%%%%%%%%%%%%%%%%%%%%%%%%%%%%%%%%%%%%%%
% Awards and Honors
\header{获奖情况}

\begin{body}
	\vspace{14pt}
	\textbf {北京邮电大学第七届ACM程序设计竞赛金奖(第四名)}  \hfill  {2013/03} \\
	\smallskip

	\textbf {北京邮电大学第六届ACM程序设计竞赛银奖(第八名)}  \hfill  {2012/03} \\
%	\smallskip
%	\textbf {IBM 大型主机技术 2011 全国应用大赛三等奖}  \hfill  {2011/10} \\

	\smallskip
	\textbf {湖南省第六届大学生程序设计竞赛二等奖(ACM 形式)}   \hfill  {2010/10}\\
	
	\smallskip
	\textbf{全国软件设计大赛决赛 C 语言组二等奖 } \hfill  {2010/08} \\
\end{body}


% \newpage{} % uncomment this line if you want to force a new page

\smallskip


%%%%%%%%%%%%%%%%%%%%%%%%%%%%%%%%%%%%%%%%%%%%%%%%%%%%%%%%%%%%%%%%%%%%%%%%%%%%%%%%
% Leadership
\header{课外实践}

\begin{body}
	\vspace{14pt}
	\emph {北京邮电大学腾讯创新俱乐部技术部核心成员} \hfill {} \emph{2011/09--至今}\\
	\emph {爱好户外徒步,骑行} \hfill {} \\

	\smallskip
	\emph {\textbf{本科期间:}}{} {09 年和同学一起创立第一个湖南省省内高校专注于开源理念的协会,担任技术部部长,参与举办多个开源知识讲座,在校内推广开源理念.} \hfill{} \emph{2009 - 2011}\\

\end{body}

\smallskip
\end{CJK}

\end{document}

